\documentclass[12pt]{article}
\usepackage[margin=1in]{geometry}
\usepackage{amsmath,amssymb}
\usepackage{graphicx}
\usepackage{booktabs}
\usepackage{natbib}
\usepackage{hyperref}
\usepackage{microtype}
\usepackage{xcolor}
\usepackage{setspace}
\doublespacing

\title{State Credit and Protocol Efficiency:\\
A Decomposition of Apparent Sub-Landauer Dissipation}

\author{Ian Todd\\
Sydney Medical School, University of Sydney, Australia\\
\texttt{itod2305@uni.sydney.edu.au}}

\date{}

\begin{document}
\maketitle

\begin{abstract}
Landauer's bound is often stated as a fixed cost per bit erased. The correct bound depends on entropy removed, which can be significantly less than the bit-count when the system or its environment carries structure. We show that apparent sub-Landauer episodes decompose into two distinct mechanisms: (i) \emph{state credit}---bias (negentropy) and correlations (mutual information) that reduce the reversible work bound and obey a conservation law; and (ii) \emph{protocol efficiency}---geometric structure in control space (thermodynamic length) that reduces irreversible dissipation but is not itself conserved. This decomposition yields a combined finite-time bound unifying information-theoretic and geometric contributions, and clarifies that anomalously low dissipation corresponds to spending accumulated state credit, not violating thermodynamic limits. We propose experimental tests in colloidal systems and feedback engines.
\end{abstract}

\noindent\textbf{Keywords:} Landauer principle, nonequilibrium thermodynamics, mutual information, thermodynamic length, stochastic thermodynamics

\section{Introduction}

The physics of information processing is often summarized by Landauer's limit: erasing one bit of information requires dissipating at least $k_\mathrm{B}T\ln 2$ of heat \citep{landauer1961}. This statement is correct only for a maximally uncertain bit in contact with a featureless thermal reservoir. The general bound is
\begin{equation}
    W_{\mathrm{rev}} \geq k_\mathrm{B}T \, \Delta S_{\mathrm{register}},
    \label{eq:landauer_general}
\end{equation}
where $\Delta S_{\mathrm{register}}$ is the entropy \emph{removed} from the information-bearing degrees of freedom \citep{bennett1982,parrondo2015}. A biased bit has less entropy; erasing it costs less.
Throughout, we take work $W>0$ as work performed \emph{on} the register (work input). When feedback yields net work output, we report it as $W_{\mathrm{out}}:=-W\ge 0$.

This entropic framing resolves apparent paradoxes. Experiments have demonstrated erasure below the ``$k_\mathrm{B}T\ln 2$ limit'' using tilted potentials \citep{berut2012} and work extraction from information using feedback \citep{toyabe2010}. These results are not violations---they are consequences of structure in the initial state or correlations with a measurement apparatus.

The purpose of this paper is to provide a clear accounting scheme for these phenomena, continuing the analysis of information-thermodynamic constraints in physical substrates developed in \citep{todd2025demon}. We distinguish two mechanisms:
\begin{enumerate}
    \item \textbf{State credit:} Bias and correlations reduce the \emph{reversible} work bound by reducing the entropy that must be removed. This is a true thermodynamic resource---creating it costs work; consuming it recovers work.
    \item \textbf{Protocol efficiency:} Geometric structure in the control landscape reduces \emph{irreversible} dissipation during finite-time operations. This is not a conserved resource but a property of the transformation path.
\end{enumerate}

\begin{figure}[ht]
    \centering
    \includegraphics[width=0.98\textwidth]{figures/fig1_credit_ledger.pdf}
    \caption{\textbf{The decomposition.} \emph{Top left:} State credit (bias + correlations) is a conserved resource that reduces $W_{\mathrm{rev}}$; spending it requires prior accumulation. \emph{Bottom left:} Protocol optimization (geometry) reduces $W_{\mathrm{irr}}$ but does not deplete---it is always available if implementable. Both mechanisms allow sub-Landauer operation, but only state credit obeys a conservation law.}
    \label{fig:ledger}
\end{figure}

\section{State Credit I: Bias}

Consider a binary register $X \in \{0,1\}$ with probability $p = P(X=1)$. Resetting to a standard state removes entropy
\begin{equation}
    H(X) = -p\log_2 p - (1-p)\log_2(1-p)
\end{equation}
bits. The minimal (reversible) work is
\begin{equation}
    W_{\mathrm{rev}} = k_\mathrm{B}T\ln 2 \cdot H(X).
    \label{eq:wrev_bias}
\end{equation}

For $p = 0.5$, this gives the familiar $k_\mathrm{B}T\ln 2$. For $p = 0.1$, the entropy is $H \approx 0.47$ bits and the cost drops to $0.47 \, k_\mathrm{B}T\ln 2$. The register's bias is not separate from the entropy---it \emph{is} the reduced entropy. No additional ``bias credit'' term is needed; the effect is already in $H(X)$.

The negentropy $H_{\max} - H(X)$, where $H_{\max} = \log_2|\mathcal{X}|$ is the register capacity, quantifies how much cheaper erasure is compared to the unbiased case. This negentropy was ``paid for'' when the bias was created---by measurement, asymmetric initialization, or prior computation.

\begin{figure}[ht]
    \centering
    \includegraphics[width=0.85\textwidth]{figures/fig2_tilted_register.pdf}
    \caption{\textbf{Bias reduces erasure cost.} Erasure cost (blue) scales with Shannon entropy, not bit-count. The complement (orange) is negentropy---the ``discount'' relative to an unbiased bit.}
    \label{fig:tilt}
\end{figure}

\section{State Credit II: Correlations}

If the register $X$ is correlated with an auxiliary system $Y$, the erasure bound tightens further. Given access to $Y$, the minimal work to reset $X$ is \citep{sagawa2010} (see \citep{delrio2011} for quantum extensions)
\begin{equation}
    W_{\mathrm{rev}} = k_\mathrm{B}T\ln 2 \cdot H(X|Y),
    \label{eq:wrev_corr}
\end{equation}
where $H(X|Y) = H(X) - I(X;Y)$ is the conditional entropy. The mutual information $I(X;Y)$ is a true thermodynamic resource:
\begin{itemize}
    \item Creating $I(X;Y)$ bits of correlation costs at least $k_\mathrm{B}T\ln 2 \cdot I(X;Y)$ in work.
    \item Consuming it (via conditional erasure or feedback) recovers at most the same amount.
\end{itemize}

The Sagawa--Ueda bound makes this precise \citep{sagawa2010,toyabe2010}:
\begin{equation}
    W_{\mathrm{out}} \leq k_\mathrm{B}T\ln 2 \cdot I(X;Y).
    \label{eq:sagawa_ueda}
\end{equation}

\textbf{State credit} is thus the sum of negentropy and accessible mutual information:
\begin{equation}
    C_{\mathrm{state}} = \bigl[H_{\max} - H(X)\bigr] + I(X;Y).
    \label{eq:state_credit}
\end{equation}
The reversible work bound becomes
\begin{equation}
    W_{\mathrm{rev}} = k_\mathrm{B}T\ln 2 \cdot \bigl[H_{\max} - C_{\mathrm{state}}\bigr].
    \label{eq:wrev_credit}
\end{equation}
Note that classically $I(X;Y) \leq H(X)$, so $C_{\mathrm{state}} \leq H_{\max}$ and $W_{\mathrm{rev}} \geq 0$ always. Work extraction (Eq.~\ref{eq:sagawa_ueda}) comes from \emph{feedback operations} that exploit correlations, not from negative erasure work. The state-credit perspective clarifies that such extraction is financed by previously established mutual information.

\begin{figure}[ht]
    \centering
    \includegraphics[width=0.85\textwidth]{figures/fig3_side_information.pdf}
    \caption{\textbf{Correlation credit.} For a uniform bit ($H(X)=1$), side information $I(X;Y)$ reduces the erasure bound (blue) and sets the extraction ceiling (orange). At $I=1$, erasure is free and full bit-to-work conversion is possible via feedback.}
    \label{fig:sideinfo}
\end{figure}

\subsection{Conservation of state credit}

State credit cannot increase without external work input. Under isothermal conditions with well-defined nonequilibrium free energy, standard nonequilibrium free energy accounting \citep{parrondo2015,seifert2012} implies that the change in credit over any protocol is bounded by the work invested:
\begin{equation}
    \Delta C_{\mathrm{state}} \leq \frac{W_{\mathrm{in}}}{k_\mathrm{B}T\ln 2}.
    \label{eq:conservation}
\end{equation}
This is a corollary of established results, not a new derivation. In isothermal settings, $k_\mathrm{B}T\ln 2 \cdot C_{\mathrm{state}}$ is the information free energy available to offset work---``credit'' is not merely a metaphor but a thermodynamic quantity. The inequality expresses the intuition that credit can be transferred between subsystems, converted between forms (measurement converts work into correlation), or dissipated---but not created from nothing. It assumes the system-plus-reservoir entropy is non-decreasing and that mutual information is explicitly tracked.

\section{Protocol Efficiency: Geometry}

Bias and correlation concern the \emph{endpoints} of a transformation (what states are involved). A separate question is \emph{how} you get there. Even between fixed initial and final distributions, different protocols incur different dissipation.

For slow driving near equilibrium, total work decomposes as
\begin{equation}
    W = W_{\mathrm{rev}} + W_{\mathrm{irr}},
\end{equation}
where $W_{\mathrm{rev}}$ is the reversible (quasistatic) work from Eq.~\eqref{eq:wrev_credit} and $W_{\mathrm{irr}}$ is irreversible dissipation. In the linear-response regime (slow driving, near equilibrium), the latter is bounded by \citep{sivak2012,crooks2007}
\begin{equation}
    W_{\mathrm{irr}} \geq \frac{k_\mathrm{B}T}{2\tau} \mathcal{L}^2,
    \label{eq:thermolength}
\end{equation}
where $\tau$ is the protocol duration and $\mathcal{L}$ is the \textbf{thermodynamic length}---the path length through parameter space measured in a generalized friction metric derived from fluctuations (often Fisher-information-related for canonical families). This bound may require corrections far from equilibrium.

Consider univariate Gaussian distributions with mean $\mu$ and standard deviation $\sigma$. The Fisher metric is
\begin{equation}
    ds^2 = \frac{d\mu^2}{\sigma^2} + \frac{2\,d\sigma^2}{\sigma^2}.
    \label{eq:fisher}
\end{equation}
This metric is \emph{anisotropic}: changes in $\mu$ are expensive when $\sigma$ is small (the distribution is narrow and ``notices'' shifts), while changes in $\sigma$ are uniformly weighted. Geodesics---paths minimizing $\mathcal{L}$---curve through this landscape to exploit cheap directions.

\begin{figure}[ht]
    \centering
    \includegraphics[width=0.85\textwidth]{figures/fig4_thermodynamic_length.pdf}
    \caption{\textbf{Protocol efficiency.} In the Fisher geometry of Gaussians, geodesics (curved) are shorter than axis-aligned protocols. For duration $\tau$, excess dissipation scales as $\mathcal{L}^2/\tau$. In this example, $\mathcal{L}_{\mathrm{axis}}/\mathcal{L}_{\mathrm{geo}} \approx 5$, so the naive protocol dissipates ${\sim}25\times$ more than the geodesic.}
    \label{fig:thermolength}
\end{figure}

\textbf{Key distinction:} Protocol efficiency is \emph{not} a conserved resource. Using an optimal protocol does not ``spend'' anything---it simply avoids waste. The geodesic is always available; the question is whether the controller can implement it. This is why we separate protocol efficiency from state credit:
\begin{itemize}
    \item State credit obeys a conservation law (Eq.~\ref{eq:conservation}).
    \item Protocol efficiency does not. It reduces $W_{\mathrm{irr}}$ but cannot make $W_{\mathrm{rev}}$ negative.
\end{itemize}

\section{Combined Bound}

Putting together state credit and protocol efficiency, the total work for a finite-time operation satisfies
\begin{equation}
    W \;\geq\; k_\mathrm{B}T\ln 2 \cdot \bigl[H_{\max} - C_{\mathrm{state}}\bigr] \;+\; \frac{k_\mathrm{B}T}{2\tau}\mathcal{L}^2.
    \label{eq:combined}
\end{equation}
The first term is the reversible bound (reduced by state credit); the second is the irreversible floor (reduced by protocol efficiency). Both must be paid, but they are paid in different currencies:
\begin{itemize}
    \item The first term approaches zero as $C_{\mathrm{state}} \to H_{\max}$ (erasure becomes free).
    \item The second term is always non-negative (dissipation cannot be negative).
\end{itemize}

Episodes of ultra-low dissipation occur when state credit is high (reducing the reversible term) \emph{and} the protocol is efficient (reducing the irreversible term). Eq.~\eqref{eq:combined} bounds \emph{work input} for finite-time state transformations; net work \emph{output} arises in feedback cycles where mutual information is converted to work (Eq.~\ref{eq:sagawa_ueda}) and must be repaid by resetting the information-bearing degrees of freedom.

\textbf{Worked example.} Consider erasing a biased bit ($p=0.1$, so $H \approx 0.47$) that is correlated with an auxiliary system ($I = 0.3$ bits). The state credit is $C_{\mathrm{state}} = (1 - 0.47) + 0.3 = 0.83$ bits, so the reversible work is $W_{\mathrm{rev}} = k_\mathrm{B}T\ln 2 \cdot (1 - 0.83) = 0.17\,k_\mathrm{B}T\ln 2$---about 17\% of the naive Landauer cost. If the protocol duration is $\tau$ (measured in units of the relevant relaxation time) with thermodynamic length $\mathcal{L} = 1.5$ (dimensionless), then Eq.~\eqref{eq:combined} implies
\[
    \frac{W}{k_\mathrm{B}T\ln 2} \gtrsim 0.17 + \frac{\mathcal{L}^2}{2\tau\ln 2}.
\]
For $\tau = 10$, the irreversible term is $\mathcal{L}^2/(2\tau\ln 2) = 2.25/(20 \times 0.693) \approx 0.16$, giving $W/(k_\mathrm{B}T\ln 2) \gtrsim 0.17 + 0.16 = 0.33$, i.e.\ $W \gtrsim 0.23\,k_\mathrm{B}T$, well below the naive $k_\mathrm{B}T\ln 2 \approx 0.69\,k_\mathrm{B}T$.

\section{Experimental Predictions}

\subsection{Colloidal erasure with variable bias}

B\'erut et al.\ \citep{berut2012} measured heat dissipation during erasure of a colloidal particle in a double-well potential. Extending their protocol:
\begin{itemize}
    \item Vary the initial trap asymmetry to create bias $p \neq 0.5$.
    \item Prediction: measured heat $Q = k_\mathrm{B}T\ln 2 \cdot H(p) + W_{\mathrm{irr}}$, where $H(p)$ is the binary entropy.
    \item At $p = 0.1$, the reversible contribution should be ${\sim}47\%$ of the symmetric case.
\end{itemize}

\subsection{Feedback engine with partial information}

Toyabe et al.\ \citep{toyabe2010} demonstrated work extraction using feedback. A more stringent test:
\begin{itemize}
    \item Introduce controlled noise into the measurement channel, reducing $I(X;Y)$ below 1 bit.
    \item Prediction: extractable work $W_{\mathrm{out}} \leq k_\mathrm{B}T\ln 2 \cdot I(X;Y)$, saturating the Sagawa-Ueda bound.
    \item The mutual information can be independently estimated from measurement statistics.
\end{itemize}

\subsection{Protocol optimization in molecular machines}

For molecular motors or enzymes operating in finite time:
\begin{itemize}
    \item Compare dissipation under naive (e.g., linear ramp) vs.\ optimized protocols.
    \item Prediction: irreversible dissipation ratio scales as $(\mathcal{L}_{\mathrm{naive}}/\mathcal{L}_{\mathrm{opt}})^2$.
    \item This is testable in optical trap experiments where the control protocol can be varied while holding endpoints fixed.
\end{itemize}

\subsection{Credit depletion in finite reservoirs}

State credit conservation (Eq.~\ref{eq:conservation}) predicts depletion effects when the credit source is finite:
\begin{itemize}
    \item Consider a register coupled to a finite biasing reservoir (e.g., a small thermal gradient or a finite-capacity battery).
    \item First erasure cycle: low dissipation (credit available from reservoir).
    \item Subsequent cycles without reservoir replenishment: dissipation increases as reservoir equilibrates.
    \item Prediction: dissipation relaxes toward unbiased-Landauer cost with a time constant set by reservoir relaxation.
\end{itemize}
This distinguishes ``cheap because structured'' from ``cheap because slow'' (the latter shows no depletion).

\section{Discussion}

\subsection{Relation to existing frameworks}

The state-credit/protocol-efficiency distinction reorganizes established results \citep{parrondo2015,seifert2012} into a cleaner accounting:

\begin{itemize}
    \item \textbf{``Beating Landauer''}: No system beats the entropic bound. Systems with low $H(X|Y)$ simply have less entropy to remove.
    \item \textbf{Maxwell's demon}: The demon's memory is a correlation reservoir. Erasing it dissipates exactly what was ``saved'' during sorting \citep{bennett1982}.
    \item \textbf{Feedback engines}: Work extraction is financed by mutual information, which was created by a prior measurement that cost at least as much.
\end{itemize}

The decomposition also connects to resource-theoretic approaches to thermodynamics, where nonequilibrium states are treated as resources that can be ``spent'' to perform otherwise-forbidden operations \citep{brandao2013}. State credit is essentially the resource content of the initial state; protocol efficiency is the quality of the resource conversion.

\subsection{Outlook: from structure to codes}

The framework presented here assumes a fixed state space. In physical systems---particularly biological ones---structure can arise dynamically through confinement and history dependence. Nonergodic dynamics generically create bias without explicit symbolic memory; when such bias becomes addressable by low-dimensional control, stable codes emerge. The credit perspective then quantifies the thermodynamic ``financing'' available for code formation and maintenance. Extending the present accounting to variable-dimensional codes and nonstationary reservoirs is a natural direction for future work.

\subsection{Limitations}

This framework is accounting, not dynamics. It does not predict:
\begin{itemize}
    \item How fast credit can be created or spent (depends on relaxation timescales).
    \item The optimal protocol for a given system (requires solving a control problem).
    \item How credit converts between forms (though conservation constrains the total).
\end{itemize}

The thermodynamic-length bound (Eq.~\ref{eq:thermolength}) holds in linear response and may require corrections far from equilibrium.

\section{Conclusion}

Landauer's bound is not a fixed tax on computation. It is a lower limit that depends on the entropy removed, which depends on the state credit available. Bias and correlations are thermodynamic resources obeying a conservation law; protocol efficiency reduces waste but is not conserved.

Episodes of ultra-low dissipation or work extraction are withdrawals from the state-credit account, not thermodynamic free lunches. The accounting clarifies what must be measured---entropy, mutual information, thermodynamic length---to predict dissipation in real systems. For information processors operating in structured environments, the question is not ``how many bits?'' but ``how much credit, and how good is the protocol?''

\bibliographystyle{unsrt}
\bibliography{references}

\end{document}
